\section{学位论文工作计划}\label{sec:schedule}
\BeforeBeginEnvironment{tabular}{\zihao{-4}}
{
\noindent
% \begin{longtable}{|c|c|c|}
\begin{longtable}{| p{0.20\textwidth}<{\centering} | p{0.40\textwidth}<{\centering} | p{0.311\textwidth}<{\centering} |}
    \hline 
	\multicolumn{1}{|c|}{起讫日期} & 	\multicolumn{1}{c}{主要完成研究内容} & 	\multicolumn{1}{|c|}{预期成果} \\
	\hline 
	\tabincell{c}{2017年09月 \\ \pozhehao \\2018年03月}   &  基础知识学习 &   \tabincell{c}{完成文献搜集与该方向基\\ 本知识储备} \\ 
	\hline 
	\tabincell{c}{2018年04月 \\ \pozhehao \\2018年06月} &  研究点1 &   完成实验 \\ 
	\hline 
	\tabincell{c}{2018年07月 \\ \pozhehao \\2018年08月} &  研究点1 &   发表论文SCI一篇 \\ 
	\hline 
	\tabincell{c}{2018年09月 \\ \pozhehao \\2018年10月} &  研究点2 &   完成实验 \\ 
	\hline 
    \tabincell{c}{2018年11月 \\ \pozhehao \\2018年12月} &  研究点2 &   发表论文EI一篇 \\ 
    \hline 
    \tabincell{c}{2019年01月 \\ \pozhehao \\2019年02月} &  研究点3 &   完成实验 \\ 
    \hline 
    \tabincell{c}{2019年03月 \\ \pozhehao \\2019年04月} &  研究点3 &   发表论文EI一篇 \\ 
    \hline 
    \tabincell{c}{2019年05月 \\ \pozhehao \\2019年06月} &  研究点4 &   完成实验 \\ 
    \hline 
    \tabincell{c}{2019年07月 \\ \pozhehao \\2019年08月} &  研究点4 &   发表论文EI一篇 \\ 
    \hline 
    \tabincell{c}{2019年09月 \\ \pozhehao \\2019年09月} &  研究点5 &   完成实验 \\ 
    \hline 
    \tabincell{c}{2019年10月 \\ \pozhehao \\2019年10月} &  研究点5 &   发表论文EI一篇 \\ 
    \hline 
	\tabincell{c}{2019年11月 \\ \pozhehao \\2020年01月} &  撰写毕业论文 &  完成毕业论文 \\ 
	\hline 
\end{longtable}
注:每个子阶段不得超过3个月;预期成果中必须包含成果的形式、数量、质量等可考性指标该计划将作为论文研究进展检查的依据。
\indent
}