\section{学位论文工作计划}\label{sec:schedule}
{
\noindent
\begin{slongtable}{| C{0.20\textwidth} | C{0.40\textwidth} | C{0.311\textwidth} |}
  \hline 
	\multicolumn{1}{|c|}{起讫日期} & 	\multicolumn{1}{c}{主要完成研究内容} & 	\multicolumn{1}{|c|}{预期成果} \\
	\hline 
	2017年09月 \par\pozhehao\par 2018年03月  & 基础知识学习 & 完成文献搜集与该方向基本知识储备 \\ 
	\hline 
	2018年04月 \par\pozhehao\par 2018年06月 & 研究点1 &  完成实验 \\ 
	\hline 
	2018年07月 \par\pozhehao\par 2018年08月 & 研究点1 &  发表论文SCI一篇 \\ 
	\hline 
	2018年09月 \par\pozhehao\par 2018年10月 & 研究点2 &  完成实验 \\ 
	\hline 
    2018年11月 \par\pozhehao\par 2018年12月 & 研究点2 &  发表论文EI一篇 \\ 
    \hline 
    2019年01月 \par\pozhehao\par 2019年02月 & 研究点3 &  完成实验 \\ 
    \hline 
    2019年03月 \par\pozhehao\par 2019年04月 & 研究点3 &  发表论文EI一篇 \\ 
    \hline 
    2019年05月 \par\pozhehao\par 2019年06月 & 研究点4 &  完成实验 \\ 
    \hline 
    2019年07月 \par\pozhehao\par 2019年08月 & 研究点4 &  发表论文EI一篇 \\ 
    \hline 
    2019年09月 \par\pozhehao\par 2019年09月 & 研究点5 &  完成实验 \\ 
    \hline 
    2019年10月 \par\pozhehao\par 2019年10月 & 研究点5 &  发表论文EI一篇 \\ 
    \hline 
	2019年11月 \par\pozhehao\par 2020年01月 & 撰写毕业论文 & 完成毕业论文 \\ 
	\hline 
\end{slongtable}
注:每个子阶段不得超过3个月;预期成果中必须包含成果的形式、数量、质量等可考性指标该计划将作为论文研究进展检查的依据。
\indent
}